\ifthenelse {\boolean{en}}
  {\section{Web Applications} \label{sec:web_app}}
  {\section{Aplicativos Web} \label{sec:web_app}}

\cvitem{}{\textbf{Easy Bike} -- \url{http://easy.bike}} \label{easy_bike}
\cvitem{}{Ruby, Python, \pand Javascript}
% \cvitem{URL}{\url{http://easy.bike}}
\cvitem{}{It finds the best cycling routes using bike-sharing systems in more than 390 cities distributed in 42 countries. It considers real-time information of how many bicycles are available closest to you and how many available bike stands are available in the stations close to the destination address. It then chooses the best stations and route for you. Besides, it provides city-specific domains such as \href{http://paris.easy.bike}{paris.easy.bike}, \href{http://london.easy.bike}{london.easy.bike}, or \href{http://ny.easy.bike}{ny.easy.bike}, etc., to directly access the service in the respective cities.}

\cvitem{}{\textbf{Proconfie} -- \url{http://www.proconfie.com}} \label{proconfie}
\cvitem{}{Ruby, Python, \pand R}
\cvitem{}{It helps people to choose companies based on historical problems presented with other customers. Received Honorary Mention award from the Brazilian Ministry of Justice. Refer to Section \hyperref[sec:awards]{Awards} for further information.}

\cvitem{}{\textbf{CEP Aberto} -- \url{http://www.cepaberto.com}}
\cvitem{}{Ruby, Python, \pand Javascript}
\cvitem{}{Collaborative application that aims to publicly open the Brazilian Postal Code (CEP) data. It currently contains information of about 1 million CEPs. It provides an API for developers willing to use geolocalized Postal Code data and, for the end-users, the possibility to edit, add, fix, follow, etc, specific postal-coded-locations in order to improve the quality of the data. About a thousand of registered users.}