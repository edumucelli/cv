\ifthenelse {\boolean{en}}
  {\section{Web and Android Applications} \label{sec:web_app}}
  {\section{Aplicativos Web e Android} \label{sec:web_app}}

\cvitem{}{\textbf{Eazy.bike} -- \url{http://eazy.bike} and \href{https://play.google.com/store/apps/details?id=com.eazybike.lite}{Android app}} \label{easy_bike}
\cvitem{}{Ruby, Python, Java (Android), \pand Javascript}
% \cvitem{URL}{\url{http://easy.bike}}
\cvitem{}{It finds the best cycling routes using bike-sharing systems in more than 443 cities distributed in 45 countries. It considers real-time information of how many bicycles and empty bike stands are available closest to the user and to the destination address. It then chooses the best stations and route. Additionally, the Android app provides automatic redirection in case the initial destination station gets full during user's trajectory. It can be accessed by city-specific domains for improved context information, e.g., \href{http://paris.eazy.bike}{paris.eazy.bike}, or \href{http://london.eazy.bike}{london.eazy.bike}, etc.}
% \cvitem{}{It finds the best cycling routes using bike-sharing systems in more than 390 cities distributed in 42 countries. It considers real-time information of how many bicycles are available closest to you and how many available bike stands are available in the stations close to the destination address. It then chooses the best stations and route. Additionally, the Android app provides automatic redirection in case the initial destination station gets full during user's trajectory. Besides, it provides provides city-specific domains such as \href{http://paris.eazy.bike}{paris.eazy.bike}, \href{http://london.eazy.bike}{london.eazy.bike}, or \href{http://ny.eazy.bike}{ny.eazy.bike}, etc., to directly access the Web service in the respective cities.}

\cvitem{}{\textbf{Proconfie} -- \url{http://www.proconfie.com}} \label{proconfie}
\cvitem{}{Ruby, Python, \pand R}
\cvitem{}{It helps people to choose companies based on problems presented with other customers. Received Honorary Mention award from the Brazilian Ministry of Justice. Refer to Section \hyperref[sec:awards]{Awards} for further information.}

\cvitem{}{\textbf{CEP Aberto} -- \url{http://www.cepaberto.com}}
\cvitem{}{Ruby, Python, \pand Javascript}
\cvitem{}{Collaborative application that aims to publicly open the Brazilian Postal Code (CEP) data. Contains information of about 1 million CEPs. It provides an API for developers and, for the end-users, a colaborative platform to improve the quality of the data. About 30K registered users.}