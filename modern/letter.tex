%--

\newcommand{\companyName}{Priceminister}

%-----       letter       ---------------------------------------------------------
% recipient data
% \recipient{Company Recruitment team}{Company, Inc.\\123 somestreet\\some city}
\recipient{\companyName{} team}{}
\date{\today}
\opening{Dear Sir or Madam,}
\closing{Yours faithfully,}
% \enclosure[Attached]{curriculum vit\ae{}}          % use an optional argument to use a string other than "Enclosure", or redefine \enclname
\makelettertitle

I am a Ph.D. in Computer Science at École Polytechnique, France. My Thesis tackled characterization, planning, and deployment problems of urban wireless networks using large-scale datasets of human mobility and data traffic. Besides, I am Linux and Open Source enthusiast for 15 years now. I have \hyperref[sec:open_source]{several projects}, created on my spare time.

Furthermore, I have made the Android and full-stack Web-development of \hyperref[sec:web_app]{three web applications}. E.g., from the \textit{"\href{http://eazy.bike}{Eazy.Bike}'s"} real-time back-end crawlers, R-based statistics on \textit{"\href{http://www.proconfie.com}{Proconfie}"}, to the front-end on \textit{"\href{http://www.cepaberto.com}{CEP Aberto}"}, and the Apache and MySQL fine-tuning for the three of them within the same VPS.

% Briefly, the first, \textit{"\href{http://easy.bike}{Easy.Bike}"} is a free application that helps bike-sharing cyclers to find the best bicycle stations and route from their places to their destinations. The second, \textit{"\href{http://www.proconfie.com}{Proconfie}"}, is a free application that helps people to choose companies based on historical problems presented with other customers. It received Honorary Mention award from the Brazilian Ministry of Justice. Finally, \textit{"\href{http://www.cepaberto.com}{CEP Aberto}" (Open Postal Code)} is a free and collaborative application that aims to open publicly the Postal Code geolocalized data in Brazil, which even after several popular requests, can be only accessed by those who pay an absurd amount of money for the Brazilian Post-mail service. It provides an API for developers willing to use the Postal Code data and, for the end-users, the possibility to edit, add, fix, follow, specific postal-coded-locations in order to improve the quality of the data.

% I have worked and experienced multicultural environments. I was born in Brazil, coursed part of my Master's Degree in Italy and, since 2011, I live in France. Besides Portuguese, I speak English, French, and Italian.

I was exactly looking for job in which I could apply my knowledge, enthusiasm, and workforce. I would like to do that full-time, not only during my spare time, an that is why I am applying for the Big Data Researcher position at \companyName. 

% I've already worked and created APIs, notably worked with Twitter's API when developing an open-source applet for Cairo-Dock, which goes from the OAuth authentication to the user's streaming capability. Besides, I work daily my own projects' APIs, open-source, high-level programming as Python, Ruby and Javascript.

% To meet the positions' requirements, I am fluent in Python, having done several open source contributions and projects, and Java, as the sole developer of Eazy.bike's Android App. I possess the theoretical and practical knowledge on the Information Retrieval, having coursed a homonymous discipline in the university. I have the knowledge on Debian by using it as my only OS for 7 years now.

To meet the requirements for this position, I am fluent in Python, having done several open source contributions and projects, and Java, as the sole developer of Eazy.bike's Android App. Furthermore, I have made several web applications, which attest my web development skills.

If that is from you interest, I'd be highly interested in discussing how my experiences can aggregate to \companyName.

\makeletterclosing