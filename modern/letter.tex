%--

\newcommand{\companyName}{Priceminister}

%-----       letter       ---------------------------------------------------------
% recipient data
% \recipient{Company Recruitment team}{Company, Inc.\\123 somestreet\\some city}
\recipient{\companyName{} team}{}
\date{\today}
\opening{Dear Sir or Madam,}
\closing{Yours faithfully,}
% \enclosure[Attached]{curriculum vit\ae{}}          % use an optional argument to use a string other than "Enclosure", or redefine \enclname
\makelettertitle

I am a Ph.D. in Computer Science at École Polytechnique, France. My Thesis tackled characterization, planning, and deployment problems of urban wireless networks using large-scale datasets of human mobility and data traffic. Besides, I am Linux and Open Source enthusiast for 15 years now. I have \hyperref[sec:open_source]{several projects}, created on my spare time.

Furthermore, I have made the Android and full-stack Web-development of \hyperref[sec:web_app]{three web applications}. E.g., from the \textit{"\href{http://eazy.bike}{Eazy.Bike}'s"} real-time back-end crawlers, R-based statistics on \textit{"\href{http://www.proconfie.com}{Proconfie}"}, to the front-end on \textit{"\href{http://www.cepaberto.com}{CEP Aberto}"}, and the Apache and MySQL fine-tuning for the three of them within the same VPS.

% Briefly, the first, \textit{"\href{http://easy.bike}{Easy.Bike}"} is a free application that helps bike-sharing cyclers to find the best bicycle stations and route from their places to their destinations. The second, \textit{"\href{http://www.proconfie.com}{Proconfie}"}, is a free application that helps people to choose companies based on historical problems presented with other customers. It received Honorary Mention award from the Brazilian Ministry of Justice. Finally, \textit{"\href{http://www.cepaberto.com}{CEP Aberto}" (Open Postal Code)} is a free and collaborative application that aims to open publicly the Postal Code geolocalized data in Brazil, which even after several popular requests, can be only accessed by those who pay an absurd amount of money for the Brazilian Post-mail service. It provides an API for developers willing to use the Postal Code data and, for the end-users, the possibility to edit, add, fix, follow, specific postal-coded-locations in order to improve the quality of the data.

% I have worked and experienced multicultural environments. I was born in Brazil, coursed part of my Master's Degree in Italy and, since 2011, I live in France. Besides Portuguese, I speak English, French, and Italian.

I was exactly looking for job in which I could apply my knowledge, enthusiasm, and workforce. I would like to do that full-time, not only during my spare time, an that is why I am applying for the Big Data Researcher position at \companyName. 

% I've already worked and created APIs, notably worked with Twitter's API when developing an open-source applet for Cairo-Dock, which goes from the OAuth authentication to the user's streaming capability. Besides, I work daily my own projects' APIs, open-source, high-level programming as Python, Ruby and Javascript.

% To meet the positions' requirements, I am fluent in Python, having done several open source contributions and projects, and Java, as the sole developer of Eazy.bike's Android App. I possess the theoretical and practical knowledge on the Information Retrieval, having coursed a homonymous discipline in the university. I have the knowledge on Debian by using it as my only OS for 7 years now.

To meet the requirements for this position, I am fluent in Python, having done several open source contributions and projects, and Java, as the sole developer of Eazy.bike's Android App. Furthermore, I have made several web applications, which attest my web development skills.

If that is from you interest, I'd be highly interested in discussing how my experiences can aggregate to \companyName.

% Dear Sir or Madam,

% I am a Ph.D. in Computer Science at École Polytechnique, France. My Thesis tackled characterization, planning, and deployment problems of urban wireless networks using large-scale datasets of human mobility and data traffic. Besides, I am Linux and Open Source enthusiast for 15 years now. I have several projects (cf. sections "Some Open Source Projects" and "Web and Android applications" on CV attached), created on my spare time.

% I believe that I have an excellent fit for the Data Engineer position at Heetch for the following reasons:

% For the last 3 years, during my Ph.D., I've been dealing with huge amounts of data, e.g., billions of connection sessions from millions of users from Telecom-related datasets, millions of venues-related data, and thousands of users and their mobility patterns, which are described by billions of geolocalized points. The goal was to extract the routinary behavior of human and how impacts the network. To do so, I've have created several models and a synthetic data traffic generator, strongly based on statistical methods and machine learning. I believe this experience can be essential to the need of "Implement smart models to predict future supply and demand".

% I am experienced Ruby (on Rails), Python, Java (Android) and R programmer with several contributions on the Open Source projects, e.g., Cairo-Dock (Linux dock bar), Pybikes (Bike-sharing library made in Python), Rhythmbox (Gnome music player), etc.

% In one of my projects, Eazy.bike (http://eazy.bike), I use data comming from several sources in order to feed the database with the status of bikesharing systems from around the world. I am sure this experience can cope with this position's requirement to "Aggregate and play with data coming from various sources".

% I am a Brazilian, I've lived in Italy, and since 2011 I live in France. I speak Portuguese, Italian, English, and French. I've worked on startups, and big Telecom operators, I can adapt myself on the environment. What motivates me now is to work on a startup-like environment in which my daily decisions can positively and quickly impact the final product.

% If that is from you interest, I'd be highly interested in discussing how my experiences can aggregate to Heetch

% Yours faithfully,

% ====== Blablacar

% Dear Sir or Madam,

% I am a Ph.D. in Computer Science at École Polytechnique, France. My Thesis tackled characterization, planning, and deployment problems of urban wireless networks using large-scale datasets of human mobility and data traffic. Besides, I am Linux and Open Source enthusiast for 15 years now. I have several projects (cf. sections "Some Open Source Projects" and "Web and Android applications" on CV attached), created on my spare time.

% I believe that I have an excellent fit for the Data Scientist position at Blablacar for the following reasons:

% For the last 3 years, during my Ph.D., I've been dealing with huge amounts of data, e.g., billions of connection sessions from millions of users from Telecom-related datasets, millions of venues-related data, and thousands of users and their mobility patterns, which are described by billions of geolocalized points. The goal was to extract the routinary behavior of human and how impacts the network. To do so, I've have created several models and a synthetic data traffic generator, strongly based on statistical methods and machine learning. I believe this experience can be essential to the need of "building predictive machine learning models".

% I am experienced Python, and R programmer with several contributions on the Open Source projects, e.g., Cairo-Dock (Linux dock bar), Pybikes (Bike-sharing library made in Python), Rhythmbox (Gnome music player), etc. Besides, I have made in-house R packages during my Thesis.

% I've presented my research results in several conferences and talks (cf. https://scholar.google.fr/citations?user=pD4KMUsAAAAJ&hl=en), which attest that I can fulfill the requirement "able to explain your models clearly and to both analysts and decision makers". 

% I am a Brazilian, I've lived in Italy, and since 2011 I live in France. I speak Portuguese, Italian, English, and French. I've worked on startups, and big Telecom operators, I can adapt myself on the environment. What motivates me now is to work on a startup-like environment in which my daily decisions can positively and quickly impact the final product.

% If that is from you interest, I'd be highly interested in discussing how my experiences can aggregate to Blablacar

% Yours faithfully,

% Because I want to positively impact on people's life with my knowledge and workforce. In this context, Blablacar is a leading actor in changing how people share the resources (colaborative consumption) and provides a great service (ridesharing). Therefore, working with Blablacar's team will give me possibility to reach a positive social impact doing what I love the most.

% In order to create a synthetic mobile data traffic generator (to simulate network load traffic generated from subscribers), I've applied machine learning techniques on a dataset containing traffic information from 7 millions real subscribers. To do so, I've observed the routinary behaviour and observed identical usage patterns. This lead me to classify (hierarchichal clustering + several stopping rules implemented in R, to avoid a-priori inference of the number of clusters in order to generate a coherent number of clusters, then supervised learning algorithms) the subscribers in distinct profiles according to their usage pattern. The distance metrics were  interval-arrival, volume of traffic, and the number of traffic sessions each of the subscribers performed. Using statistical methods, e.g., bhattacharyya distance, I was able to show that the synthetic trace generated by the proposed model consistently imitates the original dataset.

% With my background on routinary behavior characterization, I would start by looking at the geospatial routinary behavior of users and (1) prototype a feature that is to offer upfront ridesharing opportunities to destinations they tend to use. Still with regards to the routinary behavior, I would (2) prototype another feature that is a driver's (possible) reminder of routinary trajectories (i.e., several instances of routine origin and destination). It might be the case that certain ridesharing drivers just forget to create the announce on the site, but actually are going to ride from the usual origin to the usual desination. Additionally, by matching users that are looking for this very trajectory currently, it is possible to incentive the driver with a-priori ridesharing participants that are already willing to share the same trajectory. A minor addition to (2) is, by using the previously trajectories created by the driver, it would be possible to, by few clicks, (re)create the up-to-date ridesharing announce.

\makeletterclosing
