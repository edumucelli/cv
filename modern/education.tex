\section{Education}
\cventry{\october 2011 --- \may 2015}{\protect\cc}{\protect\ecole}{\protect\paris}{\textit{\protect\phd}}{}  % arguments 3 to 6 can be left empty
\cventry{\march 2009 --- \july 2011}{\protect\cc}{\protect\ufmg}{\protect\mg}{\textit{\protect\master}}{}
\cventry{\august 2004 --- \august 2008}{\protect\cc}{\protect\puc}{\protect\mg}{\textit{\protect\bachelor}}{}
~\\
\cvitem{}{\textbf{Ph.D thesis}}
\cvitem{Title}{\emph{From Your Routine to Better Network Services}~\href{https://pastel.archives-ouvertes.fr/tel-01160280}{~\scriptsize\faLink}}
\cvitem{Supervisor}{Aline C. Viana~\href{mailto:aline.viana@inria.fr}{\scriptsize\faEnvelopeO}}
\cvitem{Description}{Investigated characteristics of human mobility and their impact on the network data traffic, planning and deployment. I've analyzed large-scale datasets from mobility and traffic demands generated by millions of users. Several parallelization techniques were used to summarize and assess massive amounts of data. Python multiprocessing, thread and R multi-core libraries were largely employed.}
\cvitem{Results}{5 conference papers were published, 3 journal papers are in submission, and a data traffic simulator was created.}

\cvitem{}{\textbf{Master's thesis}}
\cvitem{Title}{\emph{Centrality-based Routing for Wireless Sensor Networks}}
\cvitem{Supervisors}{Antonio A. F. Loureiro~\href{mailto:loureiro@ufmg.br}{\scriptsize\faEnvelopeO} \pand Heitor S. Ramos~\href{mailto:heitor@ic.ufal.br}{\scriptsize\faEnvelopeO}}
\cvitem{Description}{Aimed to use centrality information on the design of routing algorithms for Wireless Sensor Networks. New topological metrics were proposed, distributed algorithms to calculate them, and a tree-based routing algorithms that take advantage of them. Simulators Castalia (C++) and Sinalgo (Java) were used.}
\cvitem{Results}{1 conference and 1 journal paper published.}

\cvitem{}{\textbf{Bachelor's thesis}}
\cvitem{Title}{\emph{Data Codification, Fusion and Communication in \wsn}}
\cvitem{Supervisors}{Raquel A. F. Mini~\href{mailto:raquelmini@pucminas.br}{\scriptsize\faEnvelopeO} and Pedro O. V. Melo~\href{mailto:olmo@dcc.ufmg.br}{\scriptsize\faEnvelopeO}}
\cvitem{Description}{Aimed to create a wireless sensor's application for fire detection using a real dataset of weather measurements. A novel way to make data fusion, and codification was proposed. Simulation used NS 2 (C++ and TCL).}
\cvitem{Results}{1 conference paper published.}