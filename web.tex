\section{Full-stack, Android and iOS side-project applications} \label{sec:web_app}

% == Rosto.io
\cvitem{}{\textbf{Rosto.io} -- \url{https://www.rosto.io}} \label{rosto_io}
\cvitem{}{Python, Keras, Tensorflow, Ruby}
\cvitem{}{Full-featured face recognition SaaS. It provides face detection features such as age, gender, emotion, ethnicity, landmark, glasses, smile, hair segmentation and much more. All based on end-to-end machine learning models written in Keras and Tensorflow. It runs on top of Flask, Rails, Nginx, Docker all monitored with Prometheus and Grafana.}

% == Eazy.bike
\cvitem{}{\textbf{Eazy.bike} -- \url{http://eazy.bike}, \href{https://play.google.com/store/apps/details?id=com.eazybike.lite}{Android app}, \href{https://itunes.apple.com/app/eazy.bike/id1141095701}{iOS app}} \label{easy_bike}
\cvitem{}{Ruby, Python, Java (Android), Swift (iOS), Javascript}
\cvitem{}{Finds bike-sharing cycling routes in more than 443 cities in 45 countries. Uses real-time information of bike availability and proximity to predict the best stations to pick-up and drop-off. Mobile apps provide automatic trajectory redirection whenever destination station fills up during user's ride. City-specific domains for improved usage, e.g., \href{http://paris.eazy.bike}{paris.eazy.bike}.}
% \cvitem{}{It finds the best cycling routes using bike-sharing systems in more than 390 cities distributed in 42 countries. It considers real-time information of how many bicycles are available closest to you and how many available bike stands are available in the stations close to the destination address. It then chooses the best stations and route. Additionally, the Android app provides automatic redirection in case the initial destination station gets full during user's trajectory. Besides, it provides provides city-specific domains such as \href{http://paris.eazy.bike}{paris.eazy.bike}, \href{http://london.eazy.bike}{london.eazy.bike}, or \href{http://ny.eazy.bike}{ny.eazy.bike}, etc., to directly access the Web service in the respective cities.}

% == Proconfie.com
\cvitem{}{\textbf{Proconfie} -- \url{http://www.proconfie.com}} \label{proconfie}
\cvitem{}{Ruby, Python, R}
\cvitem{}{It helps people to choose companies based on complains presented by other customers. Received Honorary Mention award from the Brazilian Ministry of Justice. Refer to the \hyperref[sec:awards]{Awards} section for further information.}

% == Cepaberto.com
\cvitem{}{\textbf{CEP Aberto} -- \url{http://www.cepaberto.com}}
\cvitem{}{Ruby, Python, Javascript}
\cvitem{}{Collaborative application that aims to publicly open the Brazilian Postal Code (CEP) data. Contains information of about 1 million CEPs. It provides an API for developers and, for the end-users, a collaborative platform to improve the quality of the data. About 30K registered users.}