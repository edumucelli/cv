\section{Education}
\cventry{\october 2011 --- \may 2015}{\protect\cc}{\protect\ecole}{\protect\france}{\textit{\protect\phd}}{}  % arguments 3 to 6 can be left empty
\cventry{\march 2009 --- \july 2011}{\protect\cc}{\protect\ufmg}{\protect\brazil}{\textit{\protect\master}}{}
\cventry{\august 2004 --- \august 2008}{\protect\cc}{\protect\puc}{\protect\brazil}{\textit{\protect\bachelor}}{}
~\\
\cvitem{}{\textbf{Ph.D thesis}}
\cvitem{Title}{\emph{From Your Routine to Better Network Services}~\href{https://pastel.archives-ouvertes.fr/tel-01160280}{\scriptsize\faLink}}
\cvitem{Supervisor}{Aline C. Viana~\linktoemail{aline.viana@inria.fr}}
\cvitem{Description}{Investigated characteristics of human mobility and their impact on the network data traffic, planning and deployment. I've analyzed large-scale datasets from mobility and traffic demands generated by millions of users. Python's multiprocessing, thread and R multi-core libraries were used to summarize and assess massive amounts of data.}
\cvitem{Results}{Published 5 conference papers, 2 journal papers, and created a data traffic simulator.}

\ifthenelse {\boolean{long}} {

    \cvitem{}{\textbf{Master's thesis}}
    \cvitem{Title}{\emph{Centrality-based Routing for Wireless Sensor Networks}}
    \cvitem{Supervisors}{Antonio A. F. Loureiro~\linktoemail{loureiro@ufmg.br} \pand Heitor S. Ramos~\linktoemail{heitor@ic.ufal.br}}
    \cvitem{Description}{Aimed to use centrality information on the design of routing algorithms for Wireless Sensor Networks. Proposed new topological metrics, distributed algorithms to calculate them, and a tree-based routing algorithms that take advantage of them. Simulators Castalia (C++) and Sinalgo (Java) were used.}
    \cvitem{Results}{1 conference and 1 journal paper published.}

    \cvitem{}{\textbf{Bachelor's thesis}}
    \cvitem{Title}{\emph{Data Codification, Fusion and Communication in \wsn}}
    \cvitem{Supervisors}{Raquel A. F. Mini~\linktoemail{raquelmini@pucminas.br} and Pedro O. V. Melo~\linktoemail{olmo@dcc.ufmg.br}}
    \cvitem{Description}{Created a wireless sensor's application for fire detection using real dataset of weather measurements. Proposed a novel way to fuse and codify data. Simulation used NS 2 (C++ and TCL).}
    \cvitem{Results}{1 conference paper published.}

}{}